\chapter{Conclusions}\label{ch:conclusions}

An Occurrence Graph Grammar (OGG) is a way of representing the semantics of a graph grammar as a graph grammar itself. OGGs were previously defined by~\cite{Ribeiro1996} and~\cite{Corradini1996}. Notwithstanding, its original definitions do not consider Negative Application Conditions, which are nowadays essential do the modelling of complex, real-like systems.

  In our work, we proposed an extension to the framework of occurrence graph grammars in order to include Negative Application Conditions (NACs), which has not been done so far. Additionally, the process of constructing occurrence graph grammars was implemented on Verigraph, a software specification and verification system based on graph rewriting.%, which now supports doubly-typed graph grammars in general.

  %As the main objective of this thesis, we used occurrence graph grammars and its relations to generate test cases, also using Verigraph. This functionality aims to improve the quality of the software testing activity by providing test cases and oracles that are complete (regarding the point of view of rules application ordering) and compact.

In order to implement the techniques presented in this work, we took advantage of the previous work already existent in Verigraph, mainly based on categorial notions. In this sense, one of our collateral contributions was the implementation of more basic categorial operations than originally present, such as coequalizers, coproducts and colimits. This operations may now be used by different category implementations with relatively little effort. 

Besides that, Verigraph is now (as far as we know) the only tool in this field that supports the construction and analysis of doubly-typed graph grammars and occurrence graph grammars, both with and without NACs.

    Furthermore, Verigraph is a free and open source software, publicly available at github. Thus, we expect that any users in the community of graph grammars and category theory in general can find it and use it, besides making suggestions and even implementing new features according to their specific needs.

We also provided some insight about how the semantic of graph grammars, in the form of occurrence graph grammars, can be used in order to generate sets of test cases to the system under modelling, a strategy we intend to refine in our future work.

\section{Open Questions and Future Work}

Although our main objectives were accomplished during the work of this thesis, there are several other open paths that could (should) be investigated. 

%\begin{itemize}
  %\item Investigate Conflicts, Dependencies and Local Church-Rosser with Graph Constraints
  %\item Investigate whether there are a better algorithm than backtracking to compute occurrence relations

%\end{itemize}

\textbf{Basic categorial operations:} One of the features Verigraph provides is to support for different categories and analysis, nonetheless there are basic operations and different categories and types of categories which are not contemplated by the current state of Verigraph.

One of such basic features would be the dual operations of cocomplete categories, such as equalizers, products, pullbacks and limits, implemented in a \code{Complete} type class so it can be reused in the same manner as the \code{Cocomplete} one.

Also, Verigraph does not support categories that have different morphism classes besides mono-, epi- and isomorphisms. In particular, some categories are adhesive with respect to a subclass $N$ of monomorphisms or that restrict which kind of morphisms are allowed in productions.

\textbf{Incremental NACs:} The entire process of calculating the occurrence graph grammars and the later generation of test cases depends on all the NACs of the input grammar being incremental. Despite incremental NACs being sufficient for most applications and that our case studies only used grammars which respect this restriction, there may be cases where it is needed to use grammars with general NACs.

  The current implementation of Verigraph assumes the input grammars have incremental NACs only, therefore it remains as a future work to implement the algorithm that compiles arbitrary NACs to incremental ones and use it as a previous step to our main work.

\textbf{NACs in strongly safe grammars:} In our work, we defined NACs in strongly safe grammars that are only single-typed. However, the definition of doubly-typed NACs seems to be useful if one wants to point concretely each element in the core graph that triggers the NAC of an action.

The idea is to create a doubly-typed NAC for each concrete triggering of the original NAC over the core graph. Moreover, this translation may yield the creation of (possibly) many doubly-typed NACs for each original single-typed. Thus, it remains as future work to formally define this other kind of NAC, to verify whether and when it would really be useful, how to use it to improve the expressiveness for test cases generation and then implement it on Verigraph.

%\important{Test generation with non-deterministic occurrence graph grammars with nacs}

\textbf{Different graph rewriting approaches:} Occurrence graph grammars were originally defined for DPO and SPO approach without NACs, our extension adds NACs to the DPO approach. It remains open how to extend them for SPO or even other different approaches, such as SqPO and AGREE.

\textbf{Complexity of finding a total ordering:} To this point, it is not clear to us what is the complexity of an algorithm to find (or to check if it is possible to find) a total ordering of actions of a strongly safe grammar that respects both the concrete occurrence relation and arbitrary abstract restrictions.

  Although in our practical applications so far we did not find strongly safe grammars with many constraints (or any constraint at all) in the occurrence relation restriction, it is not (theoretically) difficult to create scenarios with other applications and grammars where this situation happens. Therefore, more study is necessary regarding this aspect, specifically, we want to be sure whether it is possible to find total orderings for arbitrary strongly safe graph grammars or at least under which conditions
  (besides an empty set of abstract restrictions) would it still be feasible. After that, we will know if this process is suitable to generate tests for grammars the model arbitrary systems and not only those described by use cases.

\textbf{Concurrent Graphs:} In the original definitions of Occurrence Graph Grammars, there exist the concept of concurrent graphs, which consist of all reachable graphs while executing the underlying grammar. Such a concept was not used, therefore not defined in our extension for DPO approach with NACs. Refining our work to include this concept would both complete our definitions regarding the previous works and improve the generation of tests, by allowing us to know which (intermediary) states the system can or can not assume.

\textbf{Graphical Interface:} Regarding input and output, Verigraph does not have an operational graphical user interface (GUI) yet, using the AGG tool and its \code{.ggx} file format for providing this operations. However, AGG does not support doubly-typed graph grammars nor NACs under this framework, thus the graphical visualization of the output of the test generation process is not completely possible, which makes the development of a GUI dedicated to Verigraph a necessary step to address this issue\footnote{the development
  of a responsive, dedicated GUI is currently one of the main focus of the Verites group}.

%\textbf{Interactive test case generation:} With respect to the test case generation, the GUI will also be helpful to the guided generation of different cases. Currently, the user must create the subsets of functionalities manually, as well as the $IO$ relations needed for them.

%With a graphical interface, the user will be able to interactively guide the system in the creation of the sets of rules and $IO$ relations. Specially in the case of the $IO$ relations, the system can indicate to the user which items in the set of rules are transitively identified, avoiding the construction of redundant relations.

%Another advantage would be that a ``checker'', in which the user could provide his/her own paths to Verigraph and the system would reply if this path is a valid execution or why it is not.

%\textbf{Automated source code generation:}
%  Finally, the generation of test cases is done in a textual manner, thus leaving to the test designer the responsibility of translating it into real source code. A very interesting research would be how to translate it directly to source code of specific programming languages and/or test frameworks.
