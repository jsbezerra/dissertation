\chapter{Conclusions}\label{ch:conclusions}

This thesis has the following major contributions:

\begin{itemize}
  \item extension of the framework of occurrence graph grammars to deal with negative application conditions
  \item the implementation of the calculation of occurrence graph grammars from graph grammars on verigraph
  \item the use of occurrence graph grammars and its relations to generate test cases, also using verigraph.
\end{itemize}

and collateral contributions:

  In order to implement the techniques presented in this work, it was necessary to extend the previous work already existent in verigraph. In this sense, one of the main contributions was the implementation of a generic architecture for basic categorial operations, which can be used by different category implementations with relatively little effort.

    Furthermore, verigraph is a free and open source tool, publicly available at github. Thus, we expect that any users in the community of graph grammars and category theory in general can find and use it, besides making suggestions and even implementing new features according to their specific needs.

\section{Future Work}

\begin{itemize}
  %\item Investigate Conflicts, Dependencies and Local Church-Rosser with Graph Constraints
  \item NACs over concrete elements of the core graph and not only over the type graph. 
  \item Investigate whether there are a better algorithm than backtracking to compute occurrence relations
  \item GUI for presenting Doubly-Typed Graph Grammars

\end{itemize}

  To this point, it is not clear to us what is the complexity of an algorithm to find (or to check if it is possible to find) a total ordering of actions of a strongly safe grammar that respects both the occurrence relation and arbitrary abstract restrictions.

  Although for our practical applications so far it is unlikely that the strongly safe grammars have any restriction at all, given that grammars generated from use cases usually have an existential relation that will avoid this problem, it is not difficult to come up with other applications and grammars where this situation happens.

  Therefore, more study is necessary in which 

 The entire process of calculating the occurrence graph grammars and the later generation of test cases depends on all the NACs of the input grammar being incremental. Despite incremental NACs being sufficient for most applications and that our case studies only used grammars which respect this restriction, there may be cases where it is needed to use grammars with general NACs.

  Verigraph current implementation assumes the input grammars have incremental NACs only, therefore it remains as a future work to implement the algorithm that compiles arbitrary NACs to incremental ones and use it as a previous step to our main work.
