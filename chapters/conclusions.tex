\chapter{Conclusions}\label{ch:conclusions}

This thesis has the following major contributions:

\begin{itemize}
  \item extension of the framework of occurrence graph grammars to deal with negative application conditions
  \item the implementation of the calculation of occurrence graph grammars from graph grammars on verigraph
  \item the use of occurrence graph grammars and its relations to generate test cases, also using verigraph.
\end{itemize}

and collateral contributions:

  In order to implement the techniques presented in this work, it was necessary to extend the previous work already existent in verigraph. In this sense, one of the main contributions was the implementation of a generic architecture for basic categorial operations, which can be used by different category implementations with relatively little effort.

    Furthermore, verigraph is a free and open source tool, publicly available at github. Thus, we expect that any users in the community of graph grammars and category theory in general can find and use it, besides making suggestions and even implementing new features according to their specific needs.

\section{Future Work}

\begin{itemize}
  \item Investigate Conflicts, Dependencies and Local Church-Rosser with Graph Constraints
  \item NACs over concrete elements of the core graph and not only over the type graph. 
  \item Compile general NACs to incremental NACs
  \item Investigate whether there are a better algorithm than backtracking to compute occurrence relations
  \item GUI for presenting Doubly-Typed Graph Grammars
\end{itemize}
