\chapter{Related Work}\label{ch:related-work}

Graph Transformation Systems have a wide range of applications in Computer Science, particularly in the field of model driven engineering~\cite{modelDriven}. They can be used not only to model and simulate the execution of systems~\cite{Ehrig2009}, but also to analyse and prove properties about their behaviour~\cite{Lambers},~\cite{dissertacaoDoCarinha}, ~\cite{Penneman}, as a formal basis for translating a problem between different domain or layers ~\cite{tripleGraphGrammars} and even describe their own evolution~\cite{Machado2012}.

Not only they have the benefit of a solid formalism, they also enjoy several tools and practical approaches which have been developed over the last years in order that they can be applied over real, complex software systems, such as~\cite{AGG}, ~\cite{Groove},~\cite{Henshin},~\cite{GrGen.Net}.

Our work aims to develop a particular field in which graph transformations systems can be applied in this context: model-based testing. Therefore, we present in the following some related work in the area and how they relate or differ from our own.

We found in the literature two main techniques by which tests can be generated from graph grammars: the first one is based on the unfolding of graph transformations~\cite{Baldan2004} while the second uses visual contracts as graph rules and static analysis over them.

\section{Generating tests by the unfolding of graph transformations}

This approach was first proposed to generate tests for code generators, which are tools used to translate graphical specifications of systems directly into executable code. According to \cite{Baldan2004}, these code generators are widely used in the development of embedded software, e.g. in the automotive sector, however they lack the maturity and testing when compared to compilers of standard programming languages.

According to the authors, one of the biggest problems in testing code generators is the difficulty to describe the transformation rules from a graphical model to a target language (as well as the interactions amongst the rules) in a precise and formal way. Therefore, they propose a graph transformation based approach for systematically deriving test cases in this particular scenario.

Their approach is based on the use of unfolding of graph transformation systems~\cite{Ribeiro1996} over two graph grammars, a \textit{generating grammar}, responsible for generating all possible input models, and an \textit{optimising grammar}, which formalises specific transformation steps towards code optimisation.
The main purpose is to test the optimisation phase, in an attempt to ensure no mistakes will be introduced by improving the code while preserving its behaviour.

The unfolding of graph transformations consists of\ldots

In this environment, a test case for a subset of optimising rules $R$ is an input model $G$ such that all the rules in $R$ can be computed over $G$. This is similar to our approach where, for a sequence to be executable, all rules in that sequence must be applicable over the occurrence graph. However, we also use the non-executable ones in order to generate test cases where the system is indeed expected to fail.

In spite of using the double-pushout approach the same way we do, they use the category of (typed-)hypergraphs with some restrictions, such as: isolated nodes\footnote{ Nodes that are neither the source nor the target of any edge.} can never exist in the $LHS$ or in any $NAC$ of a rule, every node that by any means becomes disconnected is considered ``garbage collected'', and NACs can extend the match only with one edge (which, according to the authors, makes them weaker than general NACs).
Thus, despite of several similarities, our results are not fully comparable.

Their approach is focused on testing the optimization steps of code generation, but little is presented about the code generation itself. Also, although their approach was proposed for a very practical application, no supporting tool was presented.


%Advantages: test of non-deterministic systems. While our approach can probably be extended to test non-deterministic systems, no attempts 
%acyclic graphs and maximal depth of the unfolding

\section{Visual Contracts}

An approach proposed by~\cite{Heckel2011},~\cite{Khan2012},~\cite{Khan2012a},~\cite{Runge2013} focusing mainly on generating test cases for service-oriented or component-based systems. Given that systems of this kind often hide their implementation, the authors use interface descriptions known as visual contracts\footnote{ Formally regarded as graph transformation rules with operation signatures}, in order to model the observable behaviour of the system.

Coverage criteria is defined by means of static analysis, where potential conflicts and dependencies amongst visual contracts are calculated and used to build a dependency graph. In this situation, despite of being called ``a dependency graph'', this structure is rather similar to our occurrence relation, summarizing the results of both conflict and dependency analysis while representing the possible orderings in which the visual contracts may be executed.

In the processes of generating test cases, it is necessary to provide also an initial graph, which is used to find out which visual contracts are applicable to it. One of such visual contracts is chosen as the first step and all the paths through the dependency graph in which each rule is applied at most once are computed and stored as a set of rule sequences. Thereafter, the sequences are enriched to encompass rules with multiple dependencies and lately redundant rules contained in larger ones are
removed. Afterwards, each sequence is executed (if possible), and any new edges in the dependency graph reached by them are added to coverage. The entire process is then repeated as long as the coverage shows improvements. 

In comparison to our work, this approach has both advantages and limitations. As an example of the first, there are: the possibility to work with attributed typed graph transformation systems and multi-rules. As for the second: it requires more user involvement during the process of test case generation, it does not enclose negative application conditions, it was planned to work in a configuration where each rule is applied at most once and although being an extension of
AGG~\cite{Taentzer2000}, the tool is not available to download.

\section{Other Tools}

Overview and Comparison with other tools

\begin{itemize}
\item AGG
\item Groove
\item Henshin
\item AutoGraph
\item Deckwerth Framework
\end{itemize}
