\chapter{Related Work}\label{ch:related-work}

\section{Semantics}

\subsection{Unfolding Semantics}

  Unfolding techniques were initially proposed for Petri Nets~\cite{Nielsen1981} and later generalized to Graph Grammars~\cite{Ribeiro1996} and other Adhesive Categories~\cite{Baldan2009}.
  The idea of Unfolding of a Graph Grammar uses that of Occurrence Graph Grammars and consists of a nondeterministic process of the grammar that expresses its behaviour: for a given grammar $G$, it is possible to build a sequence of Occurrence Grammars $O_n$ where each $O_n$ represents all computations up to depth $n$, where the depth of a concurrent computation is the length of a maximally parallel execution of the computation~\cite{Baldan2009}.

  The Unfolding begins at the start graph of the grammar $G$ and proceeds by applying all rules in all possible ways to its concurrent subgraphs, then each occurrence of a rule and each new graph element created is recorded in the unfolfding.
  As in our approach, no element is ever deleted from this structure, which can be seen as a ``partial application'' of a rule to a match, given that it generates the new items created by the application of the rule, without actually erasing the elements that have been deleted.%: intuitively, this is because such items may still be used by another production in the nondeterministic unfolding.

\subsection{Graph Processes}

\subsection{Canonical Derivations}

\section{Generating tests cases}

\subsection{Unfolding of graph transformations}

  In~\cite{Baldan2004} the unfolding of graph transformation systems is used in order to generate test cases for code generators of the automotive industry.
  Their work is based on the category of hypergraphs and, similarly to ours, it also extends their framework to include negative application conditions.

  According to the authors, code generators are widely used in the development of embedded software, however they lack the maturity and testing when compared to compilers of standard programming languages.
  Thus, one of the biggest problems in testing code generators is the difficulty to describe the transformation rules from a graphical model to a target language (as well as the interactions amongst the rules) in a precise and formal way.
  Therefore, they propose a graph transformation based approach for systematically deriving test cases in this particular scenario.

Their approach is based on the use of unfolding of graph transformation systems~\cite{Ribeiro1996} over two graph grammars, a \textit{generating grammar}, responsible for generating all possible input models, and an \textit{optimising grammar}, which formalises specific transformation steps towards code optimisation.
The main purpose is to test the optimisation phase, in an attempt to ensure no mistakes will be introduced by improving the code while preserving its behaviour.


In this environment, a test case for a subset of optimising rules $R$ is an input model $G$ such that all the rules in $R$ can be computed over $G$. This is similar to our approach where, for a sequence to be executable, all rules in that sequence must be applicable over the occurrence graph. However, we also use the non-executable ones in order to generate test cases where the system is indeed expected to fail.

In spite of using the double-pushout approach the same way we do, they use the category of (typed-)hypergraphs with some restrictions, such as: isolated nodes\footnote{ Nodes that are neither the source nor the target of any edge.} can never exist in the $LHS$ or in any $NAC$ of a rule, every node that by any means becomes disconnected is considered ``garbage collected'', and NACs can extend the match only with one edge (which, according to the authors, makes them weaker than general NACs).
Thus, despite of several similarities, our results are not fully comparable.

Their approach is focused on testing the optimization steps of code generation, but little is presented about the code generation itself. Also, although their approach was proposed for a very practical application, no supporting tool was presented.


%Advantages: test of non-deterministic systems. While our approach can probably be extended to test non-deterministic systems, no attempts 
%acyclic graphs and maximal depth of the unfolding

\subsection{Visual Contracts}

  An approach proposed by~\cite{Heckel2011},~\cite{Khan2012},~\cite{Khan2012a},~\cite{Runge2013} focusing mainly on generating test cases for service-oriented or component-based systems.
  Given that systems of this kind often hide their implementation, the authors use interface descriptions known as visual contracts\footnote{ Formally regarded as graph transformation rules with operation signatures.} in order to model the observable behaviour of the system.

Coverage criteria is defined by means of static analysis, where potential conflicts and dependencies amongst visual contracts are calculated and used to build a dependency graph. In this situation, despite of being called ``a dependency graph'', this structure is rather similar to our occurrence relation, summarizing the results of both conflict and dependency analysis while representing the possible orderings in which the visual contracts may be executed.

In the processes of generating test cases, it is necessary to provide an initial graph, which is used to find out which visual contracts are applicable to it. One of such visual contracts is chosen as the first step and all the paths through the dependency graph in which each rule is applied at most once are computed and stored as a set of rule sequences. Thereafter, the sequences are enriched to encompass rules with multiple dependencies and lately redundant rules contained in larger ones are
removed. Afterwards, each sequence is executed (if possible), and any new edges in the dependency graph reached by them are added to coverage. The entire process is then repeated as long as the coverage shows improvement.

In comparison to our work, this approach has both advantages and limitations. As an example of the first, there are: the possibility to work with attributed typed graph transformation systems and multi-rules. As for the second: it requires more user involvement during the process of test case generation, it does not enclose negative application conditions, it was planned to work in a configuration where each rule is applied at most once and although being an extension of
AGG~\cite{Taentzer2000}, the tool is not available to download.

\section{Tools}

Although Verigraph~\cite{Costa2016, verigraph, Azzi2018} is the first tool in the field to implement Occurrence Graph Grammars (with or without Negative Application Conditions), there are other Graph Grammar tools with different approaches and analysis available. In the following, we list the ones which are to some extent closely related to Verigraph.

\subsection{AGG}

  The Attributed Graph Grammar System ~\cite{Taentzer2000} is a graph transformation tool which supports typed graph grammars.
  Its main rewriting approach is the SPO, but it can configured to execute the analyses of graph grammars as in the DPO approach.
  AGG supports attributed graphs, thus elements of a graph can be enriched with algebraic types.
  This tool is focused on static analysis such as critical pair/sequence analysis and concurrent rules, but also several others like termination and consistency checking.

\subsection{Groove}The Graphs for Object-Oriented Verification (GROOVE) Tool Set~\cite{Rensink2004} aims for modelling graph grammars. As AGG, its rewriting engine also implements the SPO approach. Its main focus is the generation and exploration of state space, implementing many exploration strategies as well as an efficient search for isomorphic states. Graphs in GROOVE are untyped, however it does support labelling to simulate types in complex systems.

% Henshin
% Deckwerth Framework

\subsection{SyGrAV}

The Symbolic Graph Analysis and Verification (SyGrAV) tool prototype~\cite{Deckwerth2016} is a tool for static verification of attributed (symbolic) graph transformation systems. It is based on the DPO approach and implemented in Java. It shares with Verigraph its inspiration to maintain the source code as closely related to the theory as possible, for which it makes use of a series of APIs defining contracts over the behaviour of the underlying Categories.
