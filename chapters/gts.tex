\chapter{Graph Grammars}

\section{Overview of Graph Transformation, definitions, algorithms, etc.}

\begin{definition}[Graph] A graph is a tuple $G = (V,E,s,t)$ where: $V$ is a set of nodes, $E$ is a set of Edges and $s,t : E \rightarrow V$ are two total functions that map each edge in $E$ to its source and target in $V$.

\end{definition}

\begin{example}[Graph]
  \tinytodo{Is it necessary?}
\end{example}

\begin{definition}[Graph Morphism] Given two graphs $G_1,G_2$ with $G_i = (V_i, E_i, s_i, t_i)$ for $i$ in $[1,2]$, a graph morphism $f : G_1 \rightarrow G_2$ between them is a pair $f = (f_V,f_E)$ where $f_V : V_1 \rightarrow V_2$ and $f_E : E_1 \rightarrow E_2$ are functions that preserve the source and target functions, i.e. $f_V \circ s_1 = s_2 \circ f_E$ and $f_V \circ t_1 = t_2 \circ f_E$. \tinytodo{Talk about injective / surjective / isomorphic morphisms?} \tinytodo{Talk about morphisms composition?}
\end{definition}

\begin{definition}[Typed Graph and Typed Graph Morphism] A type graph is a distinguished graph $TG = (V_{TG},E_{TG},s_{TG},t_{TG})$ where $V_{TG}$ and $E_{TG}$ are called the node and edge type alphabets, respectively.

  A typed graph is a pair $G^T = (G, type)$ consisting of a graph $G$ and a graph morphism $type : G \rightarrow TG$.

  Given two typed graphs $G^T_1 = (G_1,type_1)$ and $G^T_2 =(G_2,type_2)$, a typed graph morphism $f : G^T_1 \rightarrow G^T_2$ is a graph morphism $f : G_1 \rightarrow G_2$ such that $type_2 \circ f = type_1$:

\diagram{
  G_1\ar[rr]^{f}\ar[dr]_{type_1} & & G_2\ar[dl]^{type_2}\\
  \ar@{}[rur]|{=}& TG &
}

\end{definition}

\begin{definition}[Positive Atomic Constraint] A \emph{positive} atomic (typed) graph constraint is of the form $PC(a)$, where $a : P \rightarrow C$ is a (typed) graph morphism. A (typed) graph $G$ satisfies $PC(a)$ if for every injective (typed) graph morphism $p : P \rightarrow G$ there is at least one injective (typed) graph morphism $q : P \rightarrow C$ such that $p = q \circ a$.

  A positive (typed) graph constraint with $a : \emptyset \rightarrow C$ is also notated $PC(C)$. Given a (typed) graph $G$, it satisfies $PC(C)$ if there is an injective (typed) graph morphism $q : C \rightarrow  G$.

\diagram{
  P\ar[rr]^{a}\ar[dr]_{p} & & C\ar[dl]^{q}\\
  & G &
}

\end{definition}

\begin{remark}\tinytodo{Talk about about how the morphism $a$ is not required to be injective, but it would not make a difference if it was?}
\end{remark}

\begin{definition}[Negative Atomic Constraint]
A \emph{negative} atomic (typed) graph constraint is of the form $NC(a)$, where $a : P \rightarrow C$ is a (typed) graph morphism. A (typed) graph $G$ satisfies $PC(a)$ if for every injective (typed) graph morphism $p : P \rightarrow G$ there is no injective (typed) graph morphism $q : P \rightarrow C$ such that $p = q \circ a$.

  A negative (typed) graph constraint with $a : \emptyset \rightarrow C$ is also notated $NC(C)$. Given a (typed) graph $G$, it satisfies $NC(C)$ if there is no injective (typed) graph morphism $q : C \rightarrow G$.

\diagram{
  P\ar[rr]^{a}\ar[dr]_{p} & & C\ar[dl]|{|}^{q}\\
  & G &
}

\end{definition}

\begin{remark} It was shown in ~\cite{Ehrig2006} that negative atomic constraints do not give more expressive power. However, we introduced this concept because it makes easier to reason about some of the purposes of this thesis in a negative rather than a positive manner.\tinytodo{related to the graph constraint definition, maybe this is not necessary}
\end{remark}

\begin{definition}[Graph Rule] A (typed) graph rule\footnote{Also called graph transformation rule or graph production} \graphrule{} is a span of injective (typed) graph morphisms \lefthand{} and \righthand{}  where the (typed) graphs $L$, $K$ and $R$ are called the left-hand side, gluing graph and right-hand side, respectively.

  Given a (typed) graph rule $p$, its inverse rule is defined by \inversegraphrule.
\end{definition}\tinytodo{Talk about match and rule applicability here}

\begin{definition}[Graph Transformation] Given a (typed) graph rule \graphrule{} and a (typed) graph $G$ with a (typed) graph morphism \match, called match, a direct (typed) graph transformation $G \xRightarrow{p,m} H$ from $G$ to a (typed) graph $H$ is a double-pushout (DPO) diagram such as:\tinytodo{Review this definition}
  
\tinytodo{Include fact about inverse graph transformation?}

\diagram{
  L\ar[d]_{m}        & & K\ar[ll]_{l}\ar[rr]^{r}\ar[d]|{k} & & R\ar[d]^{m'}\\
  G\ar@{}[urr]|{(1)} & & D\ar[ll]^{f}\ar[rr]_{g}             & & H\ar@{}[ull]|{(2)}
}
\end{definition}

\begin{definition}[Negative Application Condition] A \emph{left} negative application condition over a graph rule \graphrule{} is of the form $NAC(n)$, where \nac{} is an arbitrary \tinytodo{total?} (typed) graph morphism. A match \match{} of a rule $p$ satisfies $NAC(n)$ on $L$ (written $m \models NAC(n)$) iff $\nexists$ $q : N \rightarrow G$ with $q$ injective and $q \circ n = m$.

\diagram{
  N\ar@{.>}[dr]|{|}_{q} & L\ar[d]^{m}\ar[l]_{n}\\
   & G
}

  A match \match{} satisfies a set \mbox{$NAC_L = \{NAC(n_i)|i \in I\}$} of left $NACs$, iff \mbox{$m \models NAC(n_i)$} $\forall i \in I$.

  Analogously, a \emph{right} negative application condition over a graph rule \graphrule{} is of the form $NAC(n)$, where \rightnac{} is an arbitrary (typed) graph morphism. A comatch \comatch{} of a rule $p$ satisfies $NAC(n)$ on $R$ (written \mbox{$m' \models NAC(n)$}) iff $\nexists$ $q : N \rightarrow H$ with $q$ injective and $q \circ n = m'$\tinytodo{Make diagram of right nac?}.

  Also, a comatch \comatch{} satisfies a set \mbox{$NAC_R = \{NAC(n_i)|i \in I\}$} of right $NACs$, iff $m' \models NAC(n_i)$ $\forall i \in I$.

\end{definition}

\begin{assumption}[Left NACs] Unless stated otherwise, we will work with left $NACs$ for the rest of this thesis.\tinytodo{Explain how right nacs can be transformed into left ones}
\end{assumption}

\begin{remark}[Different NAC Satisfaction]\tinytodo{Talk about partial injective $NACs$?}
\end{remark}

\begin{definition}[Graph Transformation System and Graph Grammar] \tinytodo{Talk about the language of the grammar?} A typed graph transformation system is a pair $GTS = (TG,P)$ where $TG$ is the type graph of the system and $P$ is a set of typed graph rules with NACs.

  A typed graph grammar is a pair $GG = (GTS,S)$ where $GTS$ is a typed graph transformation system and $S$ is a typed start graph.
\end{definition}

\begin{definition}[Parallel and Sequential Independence]
\end{definition}


\section{Occurrence Graph Grammars}

\begin{definition}[Doubly-Typed Graph]
\end{definition}

\begin{definition}[Doubly-Typed Graph Morphisms]
\end{definition}

\begin{definition}[Doubly-Typed Graph Grammars]
\end{definition}

\begin{definition}[Core Graph]
\end{definition}
