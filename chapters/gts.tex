\chapter{Graph Grammars}

\section{Overview of Graph Transformation, definitions, algorithms, etc.}

\begin{definition}[Graph] A graph is a tuple $G = \left(V,E,s,t\right)$ where: $V$ is a set of nodes, $E$ is a set of Edges and $s,t : E \rightarrow V$ are two total functions that map each edge in $E$ to its source and target in $V$.

\end{definition}

\begin{example}[Graph]
  \tinytodo{Is it necessary?}
\end{example}

\begin{definition}[Graph Morphism] Given two graphs $G_1,G_2$ with $G_i = \left(V_i, E_i, s_i, t_i\right)$ for $i$ in $[1,2]$, a graph morphism $f : G_1 \rightarrow G_2$ between them is a pair $f = \left(f_V,f_E\right)$ where $f_V : V_1 \rightarrow V_2$ and $f_E : E_1 \rightarrow E_2$ are functions that preserve the source and target functions, i.e. $f_V \circ s_1 = s_2 \circ f_E$ and $f_V \circ t_1 = t_2 \circ f_E$. \tinytodo{Talk about injective / surjective / isomorphic morphisms?} \tinytodo{Talk about morphisms composition?}
\end{definition}

\begin{definition}[Typed Graph and Typed Graph Morphism] A type graph is a distinguished graph $TG = \left(V_{TG},E_{TG},s_{TG},t_{TG}\right)$ where $V_{TG}$ and $E_{TG}$ are called the node and edge type alphabets, respectively.

  A typed graph is a pair $G^T = \left(G, type\right)$ consisting of a graph $G$ and a graph morphism $type : G \rightarrow TG$.

  Given two typed graphs $G^T_1 = \left(G_1,type_1\right)$ and $G^T_2 =\left(G_2,type_2\right)$, a typed graph morphism $f : G^T_1 \rightarrow G^T_2$ is a graph morphism $f : G_1 \rightarrow G_2$ such that $type_2 \circ f = type_1$:

\diagram{
  G_1\ar[rr]^{f}\ar[dr]_{type_1} & & G_2\ar[dl]^{type_2}\\
  \ar@{}[rur]|{=}& TG &
}
  \tinytodo{Talk about he category of graphs typed over T}
\end{definition}

\begin{definition}[Positive Atomic Constraint] A \emph{positive} atomic (typed) graph constraint is of the form $PC\left(a\right)$, where $a : P \rightarrow C$ is a (typed) graph morphism. A (typed) graph $G$ satisfies $PC\left(a\right)$ if for every injective (typed) graph morphism $p : P \rightarrow G$ there is at least one injective (typed) graph morphism $q : P \rightarrow C$ such that $p = q \circ a$.

  A positive (typed) graph constraint with $a : \emptyset \rightarrow C$ is also notated $PC\left(C\right)$. Given a (typed) graph $G$, it satisfies $PC\left(C\right)$ if there is an injective (typed) graph morphism $q : C \rightarrow  G$.

\diagram{
  P\ar[rr]^{a}\ar[dr]_{p} & & C\ar[dl]^{q}\\
  & G &
}

\end{definition}

\begin{remark}\tinytodo{Talk about about how the morphism $a$ is not required to be injective, but it would not make a difference if it was?}
\end{remark}

\begin{definition}[Negative Atomic Constraint]
A \emph{negative} atomic (typed) graph constraint is of the form $NC\left(a\right)$, where $a : P \rightarrow C$ is a (typed) graph morphism. A (typed) graph $G$ satisfies $PC\left(a\right)$ if for every injective (typed) graph morphism $p : P \rightarrow G$ there is no injective (typed) graph morphism $q : P \rightarrow C$ such that $p = q \circ a$.

  A negative (typed) graph constraint with $a : \emptyset \rightarrow C$ is also notated $NC\left(C\right)$. Given a (typed) graph $G$, it satisfies $NC\left(C\right)$ if there is no injective (typed) graph morphism $q : C \rightarrow G$.

\diagram{
  P\ar[rr]^{a}\ar[dr]_{p} & & C\ar[dl]|{|}^{q}\\
  & G &
}

\end{definition}

\begin{remark} It was shown in ~\cite{Ehrig2006} that negative atomic constraints do not give more expressive power. However, we introduced this concept because it makes easier to reason about some of the purposes of this thesis in a negative rather than a positive manner.\tinytodo{related to the graph constraint definition, maybe this is not necessary}
\end{remark}

\begin{definition}[Graph Constraint] A (typed) graph constraint is a \emph{boolean} formula over atomic (typed) graph constraints, in such way that $true$, $false$ and every atomic constraint are also graph constraints. Also, if $c$ and $c_i$, with $i \in I$ for some index set $I$, are graph constraints, then $\neg c$, $\land_{i \in I} c_i$ and $\lor_{i \in I} c_i$ are also graph constraints.

  A graph $G$ satisfies a graph constraint $c$ (written $G \models c$) if:
  \begin{itemize}
    \item $c = true$
    \item $c$ is an atomic constraint and $G \models c$
    \item $c = \neg c'$ and $G \not\models c'$
    \item $c = \land_{i \in I}c_i$ and $\forall i \in I$ $G \models c_i$      
    \item $c = \lor_{i \in I}c_i$  and $\exists i \in I$ $G \models c_i$
  \end{itemize}
\end{definition}

\begin{definition}[Graph Rule]\label{def:graph-rule} A (typed) graph rule\footnote{Also called graph transformation rule or graph production.} \graphrule{} is a span of injective (typed) graph morphisms \lefthand{} and \righthand{}  where the (typed) graphs $L$, $K$ and $R$ are called the left-hand side, gluing graph and right-hand side, respectively.

  Given a (typed) graph rule $p$, its inverse rule is defined by \inversegraphrule.
\end{definition}\tinytodo{Talk about match and rule applicability here}

\begin{definition}[Graph Transformation] Given a (typed) graph rule \graphrule{} and a (typed) graph $G$ with a (typed) graph morphism \match, called match, a direct (typed) graph transformation $G \xRightarrow{p,m} H$ from $G$ to a (typed) graph $H$ is a double-pushout (DPO) diagram such as:\tinytodo{Review this definition}
  
\tinytodo{Include fact about inverse graph transformation?}

\diagram{
  L\ar[d]_{m}        & & K\ar[ll]_{l}\ar[rr]^{r}\ar[d]|{k} & & R\ar[d]^{m'}\\
  G\ar@{}[urr]|{\left(1\right)} & & D\ar[ll]^{f}\ar[rr]_{g}             & & H\ar@{}[ull]|{\left(2\right)}
}
\end{definition}

\begin{definition}[Negative Application Condition] A \emph{left} negative application condition over a graph rule \graphrule{} is of the form $NAC\left(n\right)$, where \nac{} is an arbitrary \tinytodo{total?} (typed) graph morphism. A match \match{} of a rule $p$ satisfies $NAC\left(n\right)$ on $L$ (written $m \models NAC\left(n\right)$) iff $\nexists$ $q : N \rightarrow G$ with $q$ injective and $q \circ n = m$.

\diagram{
  N\ar@{.>}[dr]|{|}_{q} & L\ar[d]^{m}\ar[l]_{n}\\
   & G
}

  A match \match{} satisfies a set \mbox{$NAC_L = \{NAC\left(n_i\right)|i \in I\}$} of left $NACs$, iff \mbox{$m \models NAC\left(n_i\right)$} $\forall i \in I$.

  Analogously, a \emph{right} negative application condition over a graph rule \graphrule{} is of the form $NAC\left(n\right)$, where \rightnac{} is an arbitrary (typed) graph morphism. A comatch \comatch{} of a rule $p$ satisfies $NAC\left(n\right)$ on $R$ (written \mbox{$m' \models NAC\left(n\right)$}) iff $\nexists$ $q : N \rightarrow H$ with $q$ injective and $q \circ n = m'$\tinytodo{Make diagram of right nac?}.

  Also, a comatch \comatch{} satisfies a set \mbox{$NAC_R = \{NAC\left(n_i\right)|i \in I\}$} of right $NACs$, iff $m' \models NAC\left(n_i\right)$ $\forall i \in I$.

\end{definition}

\begin{assumption}[Left NACs] Unless stated otherwise, we will work with left $NACs$ for the rest of this thesis.\tinytodo{Explain how right nacs can be transformed into left ones}
\end{assumption}

\begin{remark}[Different NAC Satisfaction]\tinytodo{Talk about partial injective $NACs$?}
\end{remark}

\begin{definition}[Graph Transformation System and Graph Grammar] \tinytodo{Talk about the language of the grammar?} A typed graph transformation system is a pair $GTS = \left(TG,P\right)$ where $TG$ is the type graph of the system and $P$ is a set of typed graph rules with NACs.

  A typed graph grammar is a pair $GG = \left(GTS,S\right)$ where $GTS$ is a typed graph transformation system and $S$ is a typed start graph.
\end{definition}

\section{Parallel and Sequential Independence}

One of the characteristics that make Graph Transformation Systems and Graph Grammars suitable formalisms to model and reason about parallel and/or concurrent systems is the possibility to check whether two graph rules (or two graph transformations) can be applied together at the same time (parallel independence) or in any interchangeable order (sequential independence).

In this section, we show both what it means for two graph rules (graph transformations) to be independent and how to check it. Notice that when we are reasoning about graph transformations the (in)dependence is concrete, while for the case of graph rules the (in)dependence is potential, as it would depend on a particular pattern being found on an instance graph.

\begin{definition}[Causal Dependency] Given two graph rules $p_1,p_2$ with NACs, they are \emph{causally dependent} iff for a given graph $E$ in which they interact one of the following situations occur: \tinytodo{explain and put the other morphisms in the diagram}

  \begin{enumerate}
    \item $\nexists h_{12} : R_1 -> D_2$ such that $d_2 \circ h_{12} = m_1'$
    \item $\exists! h_{12} : R_1 -> D_2$ such that $d_2 \circ h_{12} = m_1'$ but $e_2 \circ h_{12} \not\models NAC_{p_1^{-1}}$
    \item $\nexists h_{21} : L_2 -> D_1$ such that $e_1 \circ h_{21} = m_2$
    \item $\exists! h_{21} : L_2 -> D_1$ such that $e_1 \circ h_{21} = m_2$ but $d_1 \circ h_{21} \not\models NAC_{p_2}$
  \end{enumerate}

\diagram{
    N_1 & & & & N_2 & & \\
      L_1\ar[d]\ar[u]^{n_1} & K_1\ar[d]\ar[l]\ar[r] & R_1\ar[dr]^{m'_{1}} & & L_2\ar[dl]_{m_2}\ar[u]^{n_2} & K_2\ar[d]\ar[l]\ar[r] & R_2\ar[d]\\
        H_1 & D_1\ar[l]^{d_1}\ar[rr]_{e_1} & & \textit{E} & & D_2\ar[ll]^{d_2}\ar[r]_{e_2} & H_2\\
          & & & & & &
          }
\end{definition}

\begin{definition}[Conflict]\tinytodo{explain and put the other morphisms in the diagram}

  \begin{enumerate}
    \item $\nexists h_{12} : L_1 -> D_2$ such that $d_2 \circ h_{12} = m_1$
    \item $\exists! h_{12} : R_1 -> D_2$ such that $d_2 \circ h_{12} = m_1$ but $e_2 \circ h_{12} \not\models NAC_{p_1}$
    \item $\nexists h_{21} : L_2 -> D_1$ such that $d_1 \circ h_{21} = m_2$
    \item $\exists! h_{21} : L_2 -> D_1$ such that $d_1 \circ h_{21} = m_2$ but $e_1 \circ h_{21} \not\models NAC_{p_2}$
  \end{enumerate}

\diagram{
     & & N_1 & & N_2 & & \\
      R_1\ar[d] & K_1\ar[d]\ar[l]\ar[r] & L_1\ar[u]^{n_1}\ar[dr]^{e_1} & & L_2\ar[dl]_{e_2}\ar[u]^{n_2} & K_2\ar[d]\ar[l]\ar[r] & R_2\ar[d]\\
        H_1 & D_1\ar[l]^{d_1}\ar[rr]_{e_1} & & \textit{E} & & D_2\ar[ll]^{d_2}\ar[r]_{e_2} & H_2\\
          & & & & & &
          }
\end{definition}

\begin{definition}[Parallel Independence]
\end{definition}

\begin{definition}[Sequential Independence]
\end{definition}

