\chapter{Graph Processes}

\section{Motivation}

Why do we use it to generate tests?

To a more in depth-explanation, see~\cite{Ribeiro1996, Corradini2014} 

\section{Doubly-Typed Graph Grammars}

\begin{definition}[Doubly-Typed Graph] Given a type graph $T$, a \emph{doubly-typed graph} \doublyTypedGraph{} over $T$ is a tuple \doublyTypedGraph $= \left(G^T, t^{G^T},TG^T\right)$ where $G^T$ and $TG^T$ are typed graphs over $T$ and \mbox{$t^{G^T} : G^T \rightarrow TG^T$} is a typed graph morphism in \typedGraphCategory{}. We call $TG^T$ the \emph{double-type graph} and $t^{G^T}$ the double-typing morphism.

\end{definition}

\begin{definition}[Doubly-Typed Graph Morphisms]
  Given two doubly-typed graphs $G^{TG^T}$ and $H^{TG^T}$ and a graph morphism $g^T : G^T \rightarrow H^T$, we say that $g^T$ is a \emph{$TG^T$-doubly-typed graph morphism} if the following diagram weakly commutes\tinytodo{weakly commutes?}:

\diagram{
  G\ar[rr]^{g}\ar[d]_{t^{G}}& & H\ar[d]^{t^{H}} \\
  TG\ar[rr]^{id}\ar[dr]_{type_{TG}} & &TG\ar[dl]^{type_{TG}} \\
  & T &
}
\end{definition}

Notice that the type morphisms $type_G : G \rightarrow T$ and $type_H : H \rightarrow T$ can be obtained respectively as $type_{TG} \circ t^G$ and $type_{TG} \circ t^H$.

\begin{remark} Although there are different kinds of doubly-typed graph morphisms depending on whether the doubly-type graphs and the type graphs are the different, we are here only interested in the case where the doubly-typed graphs share the same double-type and type graphs. Therefore we will call \mbox{\emph{$TG^T$-doubly-typed graph morphisms}} simply by \emph{doubly-typed graph morphisms} through the rest of this work.

\end{remark}

\begin{definition}[Doubly-Typed Graph Grammars] A \emph{doubly-typed graph grammar} is a tuple $GG = \left(TG^T, I^{TG^T},P \right)$ where $TG^T$ is the double-type graph of the grammar, $I^{TG^T}$ is a doubly-typed graph corresponding to the \emph{initial graph} of the grammar and $P$ is a set of doubly-typed graph rules. 
\end{definition}

\begin{definition}[Doubly-Typed Graph Rule] Given a double-type graph $TG^T$, a doubly-typed rule with respect to $TG^T$ is a span of morphisms \doublyTypedRule{} in \doublyTypedGraphCategory{} iff $p$ is a rule in \typedGraphCategory{} (see Definition~\ref{def:graph-rule}).\tinytodo{$p$ or $p^T$?}

  Let the typing morphisms from $L^{TG^T}$, $K^{TG^T}$ and $R^{TG^T}$ be $t^{L^T}$, $t^{K^T}$ and $t^{R^T}$, respectively. For a rule $a = p^{TG^T}$ we call:

  \begin{itemize}
    \item $L_a = L_T$, $K_a = K_R$ and $R_a = R_T$, the left, gluing and right graphs of $a$.
    \item $pre_a = t^{L^T} : L^T \rightarrow TG^T$, the \emph{pre-condition} of the $a$.
    \item $post_a = t^{R^T} : R^T \rightarrow TG^T$, the \emph{post-condition} of $a$.
    \item $r_a = r^T$, the \emph{rule pattern} of $a$.\tinytodo{Not sure if we will need the rule pattern.}
  \end{itemize}
\end{definition}

\section{Relations within a Doubly-Typed Graph Grammars}

\tinytodo{We will probably need to start using incremental negative application conditions only~\cite{Corradini2013}}

\begin{definition}[Core Graph] Given a doubly-typed graph grammar \doublyTypedGraphGrammarCore{}, we have that \coreGraph{} is a \emph{core graph} iff $\forall x \in$ \coreGraph $: \exists! y \in \parens{I^T \uplus \parens{\uplus_{ai \in P} R_{ai}}}$\tinytodo{review this} such that

  \[ x =
    \begin{cases}
      in_{GG}\parens{y} & y \in I^T\\
      post_{ai}\parens{y} & y \in R_{ai} \land y \not\in rng\parens{r_{ai}}\\
    \end{cases}
   \]\tinytodo{reformulate this entire definition}

  \begin{intuition} The idea is that each element in the \emph{Core Graph} has a unique origin either being present in the initial graph or being created by a single rule.
\end{intuition}

  Each rule in a doubly-typed graph grammar whose double-type is also a core graph is called an \emph{action}. An action $a$ \emph{creates} an element $e$ iff \tinytodo{\textbf{adapt leila's thesis to dpo}}. Similarly, an action $a$ \emph{deletes} an element $e$ iff

\end{definition}

\begin{definition}[Unconditional Causal Dependency Relation]
\end{definition}

\begin{definition}[Unconditional Weak Conflict Relation]
\end{definition}

\begin{definition}[Conditional Causal Dependency Relation]
\end{definition}

\begin{definition}[Conditional Weak Conflict Relation]
\end{definition}

\begin{definition}[Occurrence Relation]
\end{definition}

\begin{definition}[Occurrence Graph Grammars]
\end{definition}
