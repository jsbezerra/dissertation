\chapter{Graph Processes}

\section{Motivation}

\section{Doubly-Typed Graph Grammars}

\begin{definition}[Doubly-Typed Graph] Given a type graph $T$, a \emph{doubly-typed graph} \doublyTypedGraph{} over $T$ is a tuple \doublyTypedGraph $= \left(G^T, t^{G^T},TG^T\right)$ where $G^T$ and $TG^T$ are typed graphs over $T$ and \mbox{$t^{G^T} : G^T \rightarrow TG^T$} is a typed graph morphism in \typedGraphCategory{}. We call $TG^T$ the \emph{double-type graph} and $t^{G^T}$ the double-typing morphism.

\end{definition}

\begin{example}[Doubly-Typed Graph Example]

\begin{figure}
  \centering
  \begin{tikzpicture}[->,>=stealth',semithick,show background rectangle]

%% Edge Types
\tikzstyle{arr}=[-latex, black, line width=0.5pt]
\tikzstyle{morph}=[-latex, black, dashed, line width=0.5pt]

%% Node Types
\tikzstyle{rect}=[rectangle, draw=black, inner sep=9pt]
\tikzstyle{circ}=[circle, draw=black, inner sep=9pt]
\tikzstyle{trigl}=[isosceles triangle, draw, shape border rotate=90, inner sep=5pt, isosceles triangle apex angle=60, isosceles triangle stretches, text depth=1ex]


\node[rect] at (1,0) (pro3) {};
\node[circ] at (4,0) (pro1) {};
\node[trigl] at (2.5,2) (pro2) {};

\draw (pro2) to node[midway,left]{} (pro1);
\draw (pro1) to node[midway,left]{} (pro3);
\draw (pro3) to node[midway,above]{} (pro2);

\end{tikzpicture}

  \caption{Doubly-Typed Graph}\label{fig:process:doubly-typed-graph}
\end{figure}

\end{example}

\begin{definition}[Doubly-Typed Graph Morphism]
  Given two doubly-typed graphs $G^{TG^T}$ and $H^{TG^T}$ and a graph morphism $g^T : G^T \rightarrow H^T$, we say that $g^T$ is a \emph{$TG^T$-doubly-typed graph morphism} if the following diagram commutes:

\diagram{
  G\ar[rr]^{g}\ar[d]_{t^{G}}& & H\ar[d]^{t^{H}} \\
  TG\ar[rr]^{id}\ar[dr]_{type_{TG}} & &TG\ar[dl]^{type_{TG}} \\
  & T &
}
\end{definition}

Notice that the type morphisms $type_G : G \rightarrow T$ and $type_H : H \rightarrow T$ can be obtained respectively as $type_{TG} \circ t^G$ and $type_{TG} \circ t^H$.

\begin{example}[Doubly-Typed Graph Morphism Example]
\end{example}

\begin{remark} There could be defined different kinds of doubly-typed graph morphisms depending on whether the doubly-type graphs and the type graphs are or not the same. However, we are here only interested in the case where all doubly-typed graphs share the same double-type and type graphs. 
  
Therefore we will call \mbox{\emph{$TG^T$-doubly-typed graph morphisms}} simply by \emph{doubly-typed graph morphisms} through the rest of this work.

\end{remark}


\begin{definition}[Doubly-Typed Graph Rule] Given a double-type graph $TG^T$, a doubly-typed rule with respect to $TG^T$ is a span of morphisms \doublyTypedRule{} in \doublyTypedGraphCategory{} iff $p$ is a rule in \typedGraphCategory{} (see Definition~\ref{def:graph-rule})\tinytodo{$p$ or $p^T$?}. 
  
  Given a doubly-typed graph rule \doublyTypedRule{} its inverse rule is defined by \inverseDoublyTypedRule{}.

  Let the double-typing morphisms from $L^{TG^T}$, $K^{TG^T}$ and $R^{TG^T}$ be $t^{L^T}$, $t^{K^T}$ and $t^{R^T}$, respectively. For a rule $a = p^{TG^T}$ we call:

  \begin{itemize}
    \item $L_a = L_T$, $K_a = K_R$ and $R_a = R_T$, the left, gluing and right graphs of $a$.
    \item $pre_a = t^{L^T} : L^T \rightarrow TG^T$, the \emph{pre-condition} of the $a$.
    \item $post_a = t^{R^T} : R^T \rightarrow TG^T$, the \emph{post-condition} of $a$.
    %\item $r_a = r^T$, the \emph{rule pattern} of $a$.\tinytodo{Not sure if we will need the rule pattern.}
  \end{itemize}
\end{definition}

\begin{example}[Doubly-Typed Graph Rule Example]
\end{example}

\begin{definition}[Negative Application Conditions on Doubly-Typed Graph Rules] A \emph{left} negative application condition over a doubly-typed graph rule $p^{TG^T}$ is of the form $NAC(n^T)$, where $n^T : L^T \rightarrow N^T$ is an arbitrary (single-)typed graph morphism. 
 
A (doubly-typed) match morphism $m^{TG^T} : L^{TG^T} \rightarrow G^{TG^T}$ of a rule $p^{TG^T}$ satisfies $NAC(n^T)$ on $L^{TG^T}$, written \mbox{$m^T \models NAC(n^T)$}, iff $\nexists$ $q^T : N^T \rightarrow G^T$ where $q^T$ injective and $q^T \circ n^T = m^T$.

\diagram{
  T & TG\ar[l]\\
  N\ar[u]\ar@{.>}[dr]|{|}_{q} & L\ar[u]\ar[d]^{m}\ar[l]_{n}\\
   & G\ar@/_2.1pc/[uu]
}

  A match $m^{TG^T} : L^{TG^T} \rightarrow G^{TG^T}$ satisfies a set \mbox{$NAC_L = \{NAC\left(n^T_i\right)|i \in I\}$} of left $NACs$, iff \mbox{$m \models NAC\left(n^T_i\right)$} $\forall i \in I$.

\emph{Right} negative application conditions are defined analogously for the right hand side of a rule and its comatch.


\end{definition}

\begin{example}[NAC Example]
\end{example}

\begin{remark}
Although we could have defined NACs whose morphisms are doubly-typed graph morphisms (which would then act specifically over doubly-typed graphs) we will not use this kind of NACs later in our work. Therefore, we will use this \emph{single-typed} NACs as the only NAC type in all of our \emph{doubly-typed graph grammars}, and $NAC(n)$ as a synonym of $NAC(n^T)$.
\end{remark}

When dealing with the general\tinytodo{classical?} definition\tinytodo{notion?} of NACs, there may occur situations where the NAC of a rule can be triggered by the cumulative effect of applying two (or more) other rules, while the same rules would not trigger this NAC without the other. This may lead to a situation where conflicts and dependencies are not stable under switch, as shown in example~\ref{ex:process:instability}.

\begin{example}[Instability of Conflicts and Dependencies]\label{ex:process:instability}

\end{example}

As we are interested in constructing a canonical representation of several possible derivations of a set of rules in a grammar, we will make use of a special type of NACs called \emph{incremental NACs}.

\emph{Incremental NACs}, originally defined in~\cite{Corradini2013} and~\cite{Corradini2014}, have the property of extending the forbidden context of a match by a single edge or a single node. This can be regarded as each NAC forbidding only one element at a time, thus there is no possible way to trigger a NAC by the cumulative effect of more than one rule\footnote{Notice that two different rules can still trigger two different incremental NACs}.\tinytodo{we don't need this definition until the conditional relations, thus should it be here?} 


\begin{definition}[Incremental NACs] Given a monomorphism \mbox{$n : L \rightarrow N$}, $NAC(n)$ is said to be incremental if for any possible pair of decompositions \mbox{$g_1 : L \rightarrow O_g;g_2 : O_g \rightarrow N$} and \mbox{$f_1 : L \rightarrow O_f;f_2 : O_f \rightarrow N$} as in the diagram below, where all morphisms are monos and $f = n = g$, there exists a mediating morphism $o_1 : O_g \rightarrow O_f$ or $o_2 : O_f \rightarrow O_g$, such that the resulting triangles
  commutes\tinytodo{explain that inc. nacs are sufficient for most applications?}.

\diagram{
  & O_g\ar[dr]^{g_2}\ar@{.>}@/_0.5pc/[dd]|<<<<<<{o_1} &  \\
  L\ar[ur]^{g_1}\ar[dr]_{f_1}\ar[rr]|<<<<{n}&     & N\\
    & O_f\ar[ur]_{f_2}\ar@{.>}@/_0.5pc/[uu]|<<<<<<<<{o_2} & 
}

\end{definition}

\begin{example}[Incremental NACs] \tinytodo{say that there is an algorithm to convert general nacs to inc, but not implemented (out of scope)?}


\end{example}

\begin{definition}[Doubly-Typed Graph Grammars] A \emph{doubly-typed graph grammar} is a tuple $GG = \left(TG^T, I^{TG^T},P \right)$ where $TG^T$ is the double-type graph of the grammar, $I^{TG^T}$ is a doubly-typed graph corresponding to the \emph{initial graph} of the grammar and $P$ is a set of doubly-typed graph rules. 
\end{definition}

\begin{example}[Doubly-Typed Graph Grammars Example]
\end{example}

\begin{definition}[Core Graph] Given a doubly-typed graph grammar \doublyTypedGraphGrammarCore{}, we have that \coreGraph{} is a \emph{core graph} iff \mbox{$\forall x \in$ \coreGraph $: \exists! y \in (I^T \uplus (\uplus_{i \in P} (R_i - K_i))$}.


\[ x =
    \begin{cases}
      in_{GG}\parens{y},$ if $y \in I^T\\
      post_i(y),$ if $y \in R_i - K_i\\
    \end{cases}
   \]\tinytodo{check whether this definition is correct/complete}

  \begin{intuition} The idea is that each element in the \emph{Core Graph} has an unique origin, either being present in the initial graph or being created by a single rule. Moreover, all elements present in the initial graph or in the rules are also present in the \emph{Core Graph}.
\end{intuition}


\end{definition}

\begin{example}[Core Graph Example]
\end{example}

\begin{definition}[Strongly Safe (Doubly-Typed) Graph Grammars] Given \doublyTypedGraphGrammarCore{} a doubly-typed graph grammar, $GG$ is said to be \emph{strongly safe} if its double-type graph is also a core graph.

  Each rule in a strongly safe graph grammar is also called an \emph{action}. We say that an action $a$ creates an element $e$ iff $e \in R(a) - K(a)$. Similarly, $a$ deletes $e$ iff \mbox{$e \in L(a) - K(a)$}.
\end{definition}

\begin{example}[Strongly Safe Graph Grammar]
\end{example}

\section{Relations within Strongly Safe Graph Grammars}

Given a strongly safe graph grammar, its core graph contains all elements used (created, read or deleted) during one possible execution of the grammar. Moreover, as each element has an unique origin within it, the Core Graph can be considered to contain the entire ``execution history'' of its underlying grammar. 

We are here interested in some of the properties that can be found by looking at this history. Particularly, which kind of relations exist among actions and elements, whether is possible to find sequences in which all the actions are applied, and which graphs can be considered valid (reachable) by this grammar.

\begin{remark}[Different Graph Transformation Interpretation] When dealing with actions, we will use a slightly different interpretation of the graph transformations:
  
First, actions are always applied over the core graph: the match of an action is equal to its pre-condition $pre_a : L \rightarrow C^T$, as well as the comatch is equal to its post-condition $post_a : R \rightarrow C^T$. 
  
  Second, elements in $pre_a - k_a$ and $post_a - k_a$ are not ``really'' deleted nor created on the core graph (since they must always be present in the core graph as it maintains the history of execution), but rather annotated as deleted or created w.r.t. to an action. Nonetheless, their creation (resp. deletion) would be performed in an execution of the underlying simply-typed grammar. Here, for the sake of simplicity, we overload these terms and continue to refer to this elements as created/deleted.

  Third, given an action $a$ with NACs and its underlying graph transformation, we restrict the search of the injective morphism $q : N \rightarrow C^T$ to the image of the match $q : N \rightarrow C^T_{L_a}$ (comatch $q : N \rightarrow C^T_{R_a}$) rather than the entire core graph. Similarly, when searching for conflicts or dependencies within two actions $a_1, a_2$ with NACs, the test of NAC satisfiability performed on the overlapping between the actions is restricted to the jointly image of the respective morphisms onto the core graph.

  The restricted to the image search is meant to avoid the triggering of NACs by elements that are present in the core graph, therefore present in the match/comatch, but that do not exist when looking only at the actions interacting at the moment. This point will be returned later on when dealing with negative application conditions\tinytodo{Do not forget to explain this in the proper session}.
%Given an action \emph{a}, we have that:
%\diagram{
%  L\ar[d]_{pre_a}        & & K\ar[ll]_{l}\ar[rr]^{r}\ar[d]|{k} & & R\ar[d]^{post_a}\\
%  C^T_{|L}\ar@{}[urr]|{\left(1\right)} & & C^T_{|K}\ar[ll]^{f}\ar[rr]_{g}             & & C^T_{|R}\ar@{}[ull]|{\left(2\right)}
%}
%\hfill %\break
\end{remark}

In~\cite{Ribeiro1996}, causal, conflict and occurrence relations for strongly safe graph grammars were defined. There, the graph transformation approach used was the \emph{Single Pushout} (SPO) without NACs. 

\cite{Corradini1996} also defined a different notion of causal relation, equivalent to the occurrence relation in the previous work, with respect to the DPO approach without NACs.

Both authors use these relations to find out whether all actions of a strongly safe graph grammar are applicable and prove some properties about it\tinytodo{which properties?}. However, this relations alone are no sufficient to our purpose when the actions may have NACs, which are nowadays essential for modelling complex systems~\cite{Corradini2014}\tinytodo{Redo this paragraph, explain better our needs}.

Here we recall these definitions, which correspond to the \emph{existential relation} and the \emph{unconditional causal dependency and conflict relations} defined in the following, and extend them to create an equivalent notion of \emph{conditional relations} that work for grammars in the DPO approach with NACs.

\begin{definition}[Existential Relation] \tinytodo{I need a better name for this relation}This is the \emph{causal relation} defined in \cite{Corradini1996} for the DPO approach without NACs. Given  \doublyTypedGraphGrammarCore{} a strongly safe graph grammar. Let actions \mbox{$a_1, a_2 \in P, a_1 \ne a_2$} and element \mbox{$e_1 \in $ \coreGraph{}}, then:

  \begin{enumerate}
    \item If $a_1$ deletes $e_1$, then $e_1 <_e a_1$.
    \item If $a_1$ creates $e_1$, then $a_1 <_e e_1$.
    \item If $a_1$ creates $e_1$ and $p_2$ preserves $e_1$, then $a_1 <_e a_2$.
    \item If $a_1$ preserves $e_1$ and $a_2$ deletes $e_1$, then $a_1 <_e a_2$. 
    \item The \emph{existential relation} $(P \cup E, \leq_e)$ is the reflexive and transitive closure of $<_e$.
  \end{enumerate}
\end{definition}

This relation represents restrictions over creation, use (preservation) and deletion of elements by actions that must be respected for the actions to be applicable. In the sense that an action $a$ must occur after all the actions that create the elements which it depends on. Similarly, $a$ must occur before all actions that delete those same elements.

\begin{example}[Existencial Relation Example]
\end{example}

In \cite{Corradini1996} it is shown that if this relation is a partial order, then all productions of the underlying grammar are applicable. However, this condition is not sufficient in grammars with NACs.

\begin{example}[Failing existencial relation with nacs]
\end{example}

The next \emph{unconditional relations} are equivalent to the \emph{causal dependency} and \emph{weak conflict} relations presented at~\cite{Ribeiro1996}, but based on the DPO, rather than the SPO approach. Moreover, \emph{conditional relations} will be defined for conflicts and dependencies with NACs.

Regarding both unconditional and conditional dependency relations, their definitions are based on the following intuition:

\begin{intuition} An action $a_1$ is a direct cause of an action $a_2$ if either $a_1$ creates some element that is needed by $a_2$ (unconditional causal dependency) or $a_1$ deletes an element that is both forbidden by a NAC of $a_2$ and existent before the application of $a_2$ (conditional causal dependency). In both cases, we have that $a_2$ can only happen after $a_1$.\tinytodo{Explain that we do not use the notion of irreversible dependency, as we are interested only in the ordered
  execution of the actions?} 
\end{intuition}

\begin{definition}[Unconditional Causal Dependency Relation\footnote{Also called Produce-Use Relation}] Given \doublyTypedGraphGrammarCore{} a strongly safe graph grammar. Let $a_1, a_2 \in P$, $a_1 \ne a_2$, \mbox{$e_1, e_2 \in $ \coreGraph{}} and $e_1 \ne e_2$. Then: 

  \begin{enumerate}
    \item The action $a_2$ is \emph{directly causally dependent} on $a_1$, written $a_1 <_{pu} a_2$, iff \mbox{$\not\exists h_{21} : L_2 \rightarrow D_1$ s.t. \mbox{$d_1 \circ h_{21} = pre_2$}}, where the two squares are pushouts and $C^T_{|R_1L_2}$ ($C^T$ restricted to the jointly-image of $post_1$ and $pre_2$) satisfies the NACs for $a_2$ and $a_1^{-1}$.

\diagram{
   & & N_1^{-1} & & N_2 & & \\
      L_1 & K_1\ar[d]\ar[l]\ar[r] & R_1\ar[u]\ar[dr]_{post_1} & & L_2\ar[u]\ar@{.>}@/_1.1pc/[dlll]|{|}_<<<<{h_{21}}\ar[dl]^{pre_2} & K_2\ar[l]\ar[r]\ar[d] & R_2\\
       & D_1\ar@{^{(}->}[rr]_{d_1} & & C^T_{|R_1L_2} & & D_2\ar@{_{(}->}[ll]^{e_2} &}

   \item The \emph{causal dependency relation between actions} $(P, \leq_{pu})$ is the reflexive and transitive closure of the direct causal dependency.
   \item The element $e_2$ is \emph{directly causally dependent} on $e_1$, written $e_1 <_{pu} e_2$, iff there is an action $a_1 \in P$ such that $a_1$ deletes $e_1$ and creates $e_2$.
   \item The \emph{causal dependency relation between elements} $(E, \leq_{pu})$ is the reflexive and transitive closure of the direct causal dependency.
   \item The \emph{unconditional causal dependency relation} of a strongly safe grammar is defined by $(P \cup E, \leq_{pu})$.
  \end{enumerate}
\end{definition}


Regarding both unconditional and conditional weak conflict relations, their definitions are based on the following intuition:

\begin{intuition} An action $a_1$ is in \emph{weak} conflict with an action $a_2$ if either $a_1$ deletes something that is needed by $a_2$ to be applied (unconditional weak conflict) or creates something that is both forbidden by a NAC of $a_2$ and not deleted before the application of $a_2$ (conditional weak conflict). In both cases, we have that $a_2$ can not happen once $a_1$ has happened, in other words: $a_2$ can only happen before $a_1$.
\end{intuition}


\begin{definition}[Unconditional Weak Conflict Relation\footnote{Also called Delete-Use Relation.}] Given \doublyTypedGraphGrammarCore{} a strongly safe graph grammar. Let $a_1, a_2 \in P$, $a_1 \ne a_2$, \mbox{$e_1, e_2 \in $ \coreGraph{}} and $e_1 \ne e_2$. Then: 

  \begin{enumerate}
    \item The action $a_1$ is in \emph{direct weak conflict} with $a_2$, written $a_2 <_{du} a_1$, iff \mbox{$\not\exists h_{12} : L_1 \rightarrow D_2$} s.t. \mbox{$d_2 \circ h_{12} = pre_1$}.

      \diagram{
          & & N_1 & & N_2 & & \\
        R_1 & K_1\ar[l]\ar[r]\ar[d] & L_1\ar[u]\ar[dr]_{pre_1} & & L_2\ar[u]\ar[dl]^{pre_2}\ar@{.>}@/_1.1pc/[dlll]|{|}_<<<<{h^{21}} & K_2\ar[l]\ar[r]\ar[d] & R_2\\
       & D_1\ar[rr]_{d_1} & & C^T_{|L_1L_2} & & D_2\ar[ll] &}
   \item The \emph{weak conflict relation between actions} $(P,\leq_{du})$ is the reflexive and transitive closure of the direct weak conflict.
   \item The element $e_2$ is \emph{in direct weak conflict} with $e_1$, written $e_2 <_{du} e_1$, iff there are actions $a_1$ and $a_2$ that respectively create $e_1$ and $e_2$ and $a_2 \leq_{du} a_1$.
   \item The \emph{weak conflict relation between elements} $(E, \leq_{du})$ is the reflexive and transitive closure of the direct weak conflict.
   \item The \emph{weak conflict relation} of a strongly safe grammar is defined by $(P \cup E, \leq_{du})$.

  \end{enumerate}
\end{definition}

It is important to recall that relationships extracted from strongly safe grammars whose actions do not have NACs are always concrete, in the sense that if an action is dependent on (conflicting with) another one, it happens because the first one creates (resp. deletes) at least one of the concrete elements necessary for the second to be applied.

On the other hand, each dependency or conflict induced by NACs may be \emph{always concrete}, \emph{non existent}, or \emph{abstract} (conditioned to a specific order of application), for example: 

Let $a_1, a_2, a_3$ be three actions of the same strongly safe grammar. Suppose that $a_1$ creates elements used by $a_2$ and $a_2$ creates elements used by $a_3$, therefore by the existential relation we know that $a_1 \leq_e a_2 \leq_e a_3$.

Now suppose that when $a_2$ is applied, it creates an element that would be forbidden by a NAC of $a_1$ and also that $a_3$ deletes this element. 

Following the classical notions of dependency and conflict with NACs, as shown in definitions~\ref{def:classic-dependency} and~\ref{def:classic-conflict}, $a_2$ would (possibly) cause a produce-forbid conflict on $a_1$ while $a_1$ would be dependent by delete-forbid on $a_3$.

However, looking at the existential relation, we can affirm that in this configuration ensures there is no conflict between $a_1$ and $a_2$, as the element that would be forbidden by the NAC of $a_1$ can only exist after $a_1$ itself was already applied. Similar reasoning is applied to the potential dependency that is identified because $a_3$ deletes the element that would be forbidden by the NAC of $a_1$\tinytodo{Draw the examples for non-existent (last paragraph), concrete and potential
dependencies/conflicts}. As a consequence, the following problem arises:

\begin{intuition}
Given actions $a_1, a_2$ in a potential delete-forbid dependency or produce-forbid conflict, we want to know when this dependency/conflict can be considered to be \emph{concrete}, \emph{non existent} or \emph{abstract}.
\end{intuition}

\begin{definition}[Conditional Causal Dependency Relation\footnote{Also called Delete-Forbid Relation}] Let \doublyTypedGraphGrammarCore{} be a strongly-safe graph grammar, where $P$ is a set of actions with incremental, non-trivially triggered NACs only.


\diagram{
  & & N^{-1}_1 & & N_2 & &\\
  L_1\ar[d] & K_1\ar[d]\ar[l]\ar[r] & R_1\ar[u]^{n_1}\ar[dr]_{post_1} & & L_2\ar[u]_{n_2}\ar@{.>}@/_1.1pc/[dlll]_<<<<{h_{21}}\ar[dl]^{pre_2} & K_2\ar[l]\ar[r]\ar[d] & R_2\\
     C^T_{|L_1D_1} & D_1\ar[rr]_{d_1}\ar[l]^{e_1} & & C^T_{|R_1L_2} & & D_2\ar[ll] &}
\hfill

  Let the dependency diagram present a potential delete-forbid dependency where $a_1$ delete from graph \coreGraph{} an element $x \in N($\coreGraph$) \cup E($\coreGraph$)$ that potentially triggers a NAC $N_2$ of $a_2$. There are three possible configurations that can determine this dependency:

\begin{enumerate}
  \item $a_2$ is somehow related to $x$ by the existential relation.
  \item $a_2$ is not related to $x$ by the existential relation, but $x$ is present in the initial graph.
  \item $a_2$ is not related to $x$ by the existential relation and $x$ is not present in the initial graph.
\end{enumerate}

  \begin{description}[style=nextline]
  \item [Related element:] If $a_2 \leq_e x$ it means that $x$ was either created by $a_2$ or by another action that must to occur after $a_2$. In this configuration the delete-forbid dependency \emph{does not exist} as the element $x$ did not exist to trigger $N_2$ before $a_2$ was applied. 

On the other hand, if $x \leq_e a_2$, then $x$ exists before $a_2$ application and must be deleted in order for $a_2$ to be applied. Since $a_1$ is the only action that deletes $x$, otherwise $GG$ would not be a strongly safe grammar, this delete-forbid is \emph{concrete} and we have that $a_1 <_{df} a_2$ .

\item [Unrelated element present on the initial graph:]
Let $x$ be not related to $a_2$ at the same time that it is not created by any action in $P$, meaning that $x$ is present on the initial graph: $x \in I^{C^T}$. In such configuration, the delete-forbid is always \emph{concrete}, given that $x$ exists before the application of $a_2$ (or any other action), preventing its application. Therefore, we have $a_1 <_{df} a_2$.

\item [Unrelated element absent from the initial graph:]
Let $x$ be not related to $a_2$, but created by a third action $a_3$, $a_1 \ne a_3 \ne a_2$ (notice that in this configuration, $a_3 <_{e} a_1$). 

Now, suppose that $a_3$ is not related to $a_2$, in which case we have an \emph{abstract} delete-forbid $a_1 <_{df} a_2$ conditioned to $a_3$: in a configuration where $a_3$ is applied before $a_2$ the produce-forbid exists, otherwise it does not. We will represent this \emph{abstract delete-forbid dependency} as $[a_1 <_{df} a_2$ | $a_3 < a_2 ]$.
  
Notice that if $a_3$ were related to $a_2$ and $a_3$ creates $x$, then $a_2$ would be related to $x$, which corresponds to the first case.
\end{description}
\end{definition}

\begin{definition}[Conditional Weak Conflict Relation\footnote{Also called Produce-Forbid Relation}] Given \doublyTypedGraphGrammarCore{} a strongly safe graph grammar, where $P$ is a set of rules with incremental, non-trivially triggered NACs only.

\diagram{
  & & N_1 & & N_2 & &\\
  R_1\ar[d] & K_1\ar[d]\ar[l]\ar[r] & L_1\ar[u]^{n_1}\ar[dr]_{pre_1} & & L_2\ar[u]_{n_2}\ar@{.>}@/_1.1pc/[dlll]_<<<<{h_{21}}\ar[dl]^{pre_2} & K_2\ar[l]\ar[r]\ar[d] & R_2\\
     C^T_{|R_1D_1} & D_1\ar[rr]_{e_1}\ar[l]^{d_1} & & C^T_{|L_1L_2} & & D_2\ar[ll] &}
\hfill

  Let $a_1,a_2 \in P$, where $a_1$ causes a potential produce-forbid conflict on $a_2$ because $a_1$ creates on an element $x \in N(C^T) \cup E(C^T)$ that may trigger a NAC $N_2$ of $a_2$. There are two possible cases to look at:

  \begin{enumerate}
    \item $a_2$ is somehow related to $x$ by the existential relation.
    \item $a_2$ is not related to $x$ by the existential relation.
  \end{enumerate}

\begin{description}[style=nextline]
  \item[Related element:]
  Let $a2 \leq_e x$, which means that $x$ is created only after $a_2$ has been applied. In such configuration, the produce-forbid does not exist as the trigger element does not occur before the application of $a_2$. 
    
  Now let $x \leq a_2$ it means that $x$ existed before $a_2$ was applied, which leads to two possible sub cases: 
  
  First, assume that there is no other action $a_3$ which deletes $x$, then the produce-forbid between $a_1$ and $a_2$ is \emph{concrete} and we have that the $a_2 <_{pf} a_1$.

    Now, assume that there exists a third action $a_3$, with $a_1 \ne a_3 \ne a_2$, which deletes $x$. This means that $a_1$ and $a_2$ are in a concrete produce-forbid conflict, but this conflict is ``annulated'' by the application of $a_3$, which must then happen before $a_2$. Notice that in this configuration we will probably have that $a_3 <_{df} a_2$\tinytodo{review this, the ``dependency'' may not occur but the order of application must contain $a_3 < a_2$}.

  \item[Unrelated element:]
    Let $a_2$ be not related to $x$ in the existential relation, then we have to possible cases.
    
    Suppose that there is no other action $a_3$ which deletes $x$, then we know for a fact that once $a_1$ has been applied and $x$ created there will no longer be possible to apply $a_2$. Therefore, this produce-forbid is \emph{concrete}, and $a_2 <_{pf} a_1$.

    On the other hand, if there is an action $a_3$ which deletes $x$ and $a_3$ is not related to $a_2$ by the existential relation, then is this configuration we have an \emph{abstract} conflict that $a_2 <_{pf} a_1$ conditioned to whether $a_3$ is applied before $a_2$. We will represent this \emph{abstract produce-forbid conflict as} \mbox{$[a_2 <_{pf} a_1$ | $a_3 < a_2]$}.
\end{description}
\end{definition}

\begin{remark}[Abstract Dependencies and Conflicts] The existence of an abstract produce-forbid conflict caused by an element $x$ is conditioned to the existence of an action which deletes $x$.

  Given an action $a_1$ which creates $x$, an action $a_2$ whose NAC forbids $x$ and provided a configuration where $a_1$ was applied before $a_2$, we have that $a_2$ can be applied only after an action $a_3$ which deletes $x$ has been applied.

  However, $a_3$ may also cause a new produce-forbid conflict on $a_2$ by creating a new element $y$ which is also forbidden by a NAC of $a_2$, on which case $a_2$ can only be applied of there is another action $a_4$ which ``turns off'' the conflict caused by $a_3$.

  In general, for each abstract produce-forbid conflict $a_2 <_{pf} a_1$ caused by an element $x$, we have that $a_2$ must be successfully applied before $a_1$ or after an action $a_{j}$ where $a_i$ deletes $x$ and $a_i \leq a_j$.

  Analogously, for each abstract delete-forbid conflict $a_1 <_{df} a_2$ caused by an element $x$, we have that $a_2$ must be successfully applied after $a_1$ or before an action $a_j$ where $a_i$ creates $x$ and $a_j \leq a_i$.
\end{remark}

\begin{definition}[Concrete Occurence Relation] The \emph{concrete occurrence relation} of a strongly safe graph grammar is defined as $(P \cup E, \leq_{co})$ where $\leq_{co}$ is the transitive and reflexive closure of $\leq_{e} \cup \leq_{du} \cup \leq_{pu} \cup$  $concrete(\leq_{df} \cup \leq_{pf})$: the union of the existential, unconditional dependency and unconditional conflict relations, which are always concrete, and the concrete conditional dependency and conflict relations.
\end{definition}

\begin{definition}[Occurence Relation Restrictions] The union of the abstract conditional dependency and abstract conditional conflict relations.
\end{definition}

\begin{definition}[Occurrence Graph Grammars] Let \doublyTypedGraphGrammarCore{} be a strongly safe graph grammar. $GG$ is an \emph{occurrence graph grammar} iff:

  \begin{enumerate}
    \item acyclic existential relation: $\forall a \in P$: $\leq_e$ is antisymmetric;
    \item acyclic concrete occurrence relation: $\forall a \in P$: $\leq_{co}$ is antisymmetric;
    \item there is at least one serialization of $a_1,\ldots,a_n \in P$ which respects the Occurrence Relation Restrictions.
  \end{enumerate}
\end{definition}

\begin{definition}[Concurrent Graphs]
\end{definition}

