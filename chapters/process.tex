\chapter{Graph Processes}

\section{Motivation}

Why do we use it to generate tests?

To a more in depth-explanation, see~\cite{Ribeiro1996} 

\section{Doubly-Typed Graph Grammars}

\begin{definition}[Doubly-Typed Graph] Given a type graph $T$, a \emph{doubly-typed graph} \doublyTypedGraph{} over $T$ is a tuple \doublyTypedGraph $= \left(G^T, t^{G^T},TG^T\right)$ where $G^T$ and $TG^T$ are typed graphs over $T$ and \mbox{$t^{G^T} : G^T \rightarrow TG^T$} is a typed graph morphism in $\cat{TGraph_T}$. We call $TG^T$ the \emph{double-type graph} and $t^{G^T}$ the double-typing morphism.
\end{definition}

\begin{definition}[Doubly-Typed Graph Morphisms]
  Given two doubly-typed graphs $G^{TG^T}$ and $H^{TG^T}$ and a graph morphism $g^T : G^T \rightarrow H^T$, we say that $g^T$ is a \emph{$TG^T$-doubly-typed graph morphism} if the following diagram weakly commutes\tinytodo{weakly commutes?}:

\diagram{
  G\ar[rr]^{g}\ar[d]_{t^{G}}& & H\ar[d]^{t^{H}} \\
  TG\ar[rr]^{id}\ar[dr]_{type_{TG}} & &TG\ar[dl]^{type_{TG}} \\
  & T &
}
\end{definition}

\begin{remark} Although there are different kinds of doubly-typed graph morphisms depending on whether the doubly-type graphs and the type graphs are the different, we are here only interested in the case where the doubly-typed graphs share the same double-type and type graphs. Therefore we will call \mbox{\emph{$TG^T$-doubly-typed graph morphisms}} simply by \emph{doubly-typed graph morphisms} through the rest of this work.

\end{remark}

\begin{definition}[Doubly-Typed Graph Grammars] A \emph{doubly-typed graph grammar} is a tuple $GG = \left( TG^T, I^{TG^T},P \right)$ where $TG^T$ is the double-type graph of the grammar, $I^{TG^T}$ is a doubly-typed graph corresponding to the \emph{initial graph} of the grammar and $P$ is a set of doubly-typed graph rules. 
\end{definition}

\begin{definition}[Core Graph]
\end{definition}

\section{Relations within a Doubly-Typed Graph Grammars}
\tinytodo{We will probably need to start using incremental negative application conditions only}

\begin{definition}{Unconditional Causal Dependency Relation}
\end{definition}

\begin{definition}{Unconditional Weak Conflict Relation}
\end{definition}

\begin{definition}{Conditional Causal Dependency Relation}
\end{definition}

\begin{definition}{Conditional Weak Conflict Relation}
\end{definition}

\begin{definition}{Occurrence Relation}
\end{definition}

\begin{definition}{Occurrence Graph Grammars}
\end{definition}
