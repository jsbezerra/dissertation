\chapter{Introduction}

Graph grammars are a suitable formalism to model complex systems in an intuitive and precise manner, providing both a graphical language and a solid formal background for systems analysis. In this framework, system states are modelled as graphs, while transitions between different states are modelled as graph transformation rules~\cite{Ehrig2006}. A graph transformation rule generically has the form $L \xRightarrow{p} R$, where there is at least one left side graph $L$ containing a pattern to be found (a match) in order for the rule to be applied over an instance graph and a right side graph $R$ corresponding to the effect of applying such a rule. Moreover, there are also additional structures that may be used as complements of rules to better tune which kinds of matches are acceptable to perform a transformation, for instance: Graph Constraints, Negative Application Conditions (NACs) and Nested Application Conditions, to cite some of the most commonly used.

There exist several approaches that might not only change the exact format of a rule, but also define different ways a rule can be applied over a given match. Only considering the domain of Category Theory there is already a handful of different approaches to graph grammars: the Single Pushout (SPO), Double Pushout (DPO), Sesqui-Pushout (SqPO), AGREE, among others. Each approach has its own advantages/disadvantages regarding the kinds of  ``operations'' they allow or forbid in the system under modelling. 
In spite of that, whichever approach is used, they all provide a way to describe the behaviour of the system together with means to analyse several properties about this behaviour, such as termination, concurrency and reachable states. Among the several analysis techniques provided by graph grammars, we have:

\begin{itemize}
  \item Critical Pair Analysis and Critical Sequence Analysis: critical pair analysis allows us to verify which rules conflict with (i.e. prohibit) the application of another and why; critical sequence analysis, which rules depend on the execution of others to be applied and why~\cite{Lambers2008a}; such analyses provide insights about the possible execution flows of the system.
  \item Calculation of Concurrent Rules: a concurrent rule summarizes in one rule the combined results of applying several different rules. In other words, it represents the combination of several different rules which can be then applied as a ``one step'' transformation rule~\cite{Lambers2008,BezerraETMF2016}.
  \item State Space Exploration and Model Checking: permits to verify, in an exhaustive fashion, whether the (graph grammar) model of a system satisfies a given specification and to prove the satisfaction of its properties~\cite{Rensink2004}.
  \item Unfolding, Graph Processes, Occurrence Graph Grammars and Canonical Derivations: they are all different means to provide the semantics of Graph Grammars, allowing us to check which graphs belong to the language of a grammar and/or which concrete derivations are possible within that grammar~\cite{Corradini1996,Ribeiro1996}.
\end{itemize}

Graph grammars have found a wide range of applications within Computer Science, specially in the field of model-driven software development, where the transformation of visual models is a vital part of the process, therefore a natural application of graph grammars~\cite{Rozenberg1997}.
As evidence of such suitability, several non-trivial systems have been modelled and studied under the optics of graph grammars, such as telephone communications~\cite{Ribeiro1996}, elevator control~\cite{Lambers2010}, railroad control~\cite{Pennemann2009} and integration of service-oriented systems~\cite{Giese2015}.
Furthermore, there are a number of software tools to support the use of graph grammars, such as AGG, a tool environment for algebraic graph transformation~\cite{Taentzer2000}; Groove, a tool for state space generation~\cite{Rensink2004} and Verigraph, a software specification and verification tool based on graph rewriting~\cite{verigraph}.

Besides its powerful applications, the use of graph grammars as a framework for modelling systems provides us with a great advantage over other formalisms: it makes it possible for non-specialists in the field to generate graph grammar models of a system and then benefit from the rigorous analyses it offers without the need for a deep understanding of its underlying theory. For example, ~\cite{Junior2015,BezerraWEIT2016,Cota2017} explain how to generate graph grammars from a set of textual requirement documents such as use cases, functional specifications and other kinds of guidelines by means of a systematic methodology. They also present guidance towards using different graph grammars analysis techniques in order to improve and verify these documents and, consequently, the systems they describe.

The field of Graph Grammars is a very active one, and researches continuously develop new ideas, such as new graph transformation approaches (e.g. Sesqui-Pushout, AGREE), analysis techniques (e.g. Essential Critical Pairs), tools (e.g. Verigraph) and ways to apply them.
Additionally, we may also benefit from the combination of already existing techniques, which is what we do in this thesis.
In our work, we combine two concepts of the Graph Grammar Theory: Occurrence Graph Grammars, defined for the SPO and DPO approaches by~\cite{Ribeiro1996, Corradini1996}, with Negative Application Conditions, defined generically by~\cite{Habel1996}, which has not been done so far.

Occurrence Graph Grammars provides a semantics for Graph Grammars encoded in structures that are also Graph Grammars themselves. 
This semantics, which tells us which graphs are part of the grammar language and which graph transformations are possible within the context of the original grammar, may be used for the analysis of the system execution in a summarized fashion and also be used for practical applications, such as test cases generation, without the necessity to use a supplementary structure or formalism.

Negative Application Conditions (NACs) are extensions of a side of a rule encoding patterns that, if found in the match (or sometimes the comatch) of a rule, forbids the transformation. In theory, they do not give any more expressive power to a graph grammar than using only rules without them~\cite{Habel1996}. In practice, they allow the modelling of systems in a much more concise and compact manner. Therefore they became really necessary in the modelling of complex, real-like systems~\cite{Corradini2013, Corradini2014}.

Our work here consists of the development of an extension for the framework of Occurrence Graph Grammars in the Double Pushout (DPO) approach in order to incorporate Negative Application Conditions, alongside with the implementation of this extension in the Verigraph system, a generic graph rewriting system based on Category Theory and written in Haskell. This implementation choice makes possible for the source code of the tool to be close to the theory domain as well as allowing other researches to implement new approaches or different models of graphs while benefiting from the already implemented techniques (as long as they conform to the categorial constructions).

%We believe that the use of graph grammars as a model for the generation of test cases and oracles may improve the reliability of the testing activity by using the solid formal semantics of the formalism, while requiring little theoretical expertise from the user. The main objective of this thesis can be summarized as follows:

%\begin{intuition}
%  \center{\textit{Given the graph grammar model of a software system, how can a set of relevant test cases and oracles be generated for the system?}}
%\end{intuition}\hfill\break

Thus, the main contributions of this thesis can be summarized as (1) the creation of an extension  to the framework of Occurrence Graph Grammars (in the DPO approach) in order to include Negative Application Conditions and (2) the implementation of this extension in Verigraph, a software specification and verification system based on graph transformations, which is now also the first tool in the field to implement the construction of Occurrence Graph Grammars for general Graph Grammars, even when considering OGGs without NACs.

\hfill \break
\textbf{Structure of the Thesis:}

\begin{description}
	\item[Chapter~\ref{ch:gts}:] In this chapter we review the basic notions of graph transformation systems, specifically under the Double-Pushout (DPO) approach. We also introduce Negative Application Conditions (NACs) and the basic notions of parallel and sequential independence of rules, which are needed for the construction of Occurrence Graph Grammars with NACs.

  \item[Chapter~\ref{ch:process}:] In this chapter we first present an overview of doubly-typed graph grammars and other concepts necessary to accomplish the construction of Occurrence Graph Grammars, as well as how these Occurrence Grammars can be used to represent the semantics of their original grammar. After reviewing these concepts, we present our extension to previous works in Occurrence Graph Grammars to include the notion of Negative Application Conditions, which is part of our thesis contribution.

  \item[Chapter~\ref{ch:verigraph}:] This chapter presents an overview of the Verigraph system, which was used to implement the techniques presented in this thesis. Verigraph in itself represents a novelty in the field of graph transformations, being the first tool in the area implemented in a functional language, which favoured its source code to be very close to the problem domain itself.

  \item[Chapter~\ref{ch:tests}:] This chapter explains in depth the step-by-step construction of an Occurrence Graph Grammar with the help of some running examples. Additionally, it demonstrates how this was implemented in Verigraph, while also providing some insight about how it can be used for test cases generation.

  \item[Chapter~\ref{ch:related-work}:] This chapter discuss related work to the one presented in this thesis. Specifically focusing on the literature about the Semantics of Graph Grammars and how to use it for test cases generation. It also lists some software tools related to Verigraph.

  \item[Chapter~\ref{ch:conclusions}:] This chapter summarizes our results and presents our conclusions. Moreover, it shows remaining open problems and future work.

  \item[Appendix~\ref{app:category-theory}:] This appendix contains a brief review of category theory and the categorial constructions used in this thesis.

  %\item[Appendix~\ref{app:use-cases}] This appendix contains the use cases and the modelled graph grammar used as a case study on section~\ref{ch:tests}.
\end{description}
