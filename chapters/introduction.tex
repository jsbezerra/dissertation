\chapter{Introduction}

\section{Overview of Software Engineering / Requirement Engineering / Tests}

\section{Formal Methods}

\section{Test covering}

\section{Objective and expected contributions}

We assume the reader has familiarity with the framework of category theory. A good introduction can be found on~\cite{Pierce1991}.

\hfill \break
\textbf{Structure of the Thesis:}

\begin{description}
  \item[Chapter~\ref{ch:gts}:] In this chapter we review the basic notions of DPO graph grammars as well as the notions of parallel and sequential independency.
  \item[Chapter~\ref{ch:verigraph}:] This chapter presents an overview of the verigraph tool, its general purpose and basic constructions. This tool was also used to implement the techniques discussed in the following chapters.
  \item[Chapter~\ref{ch:concurrent-rules}:] In this chapter we present the construction of concurrent rules in addition to the problems that may arise from their calculation. Moreover, we present techniques that can be used to work around the problems.
  \item[Chapter~\ref{ch:process}:] In this chapter we present an overview of doubly-typed graph grammars and other constructions necessary to accomplish occurrence graph grammars, and how occurrence graph grammars can be used to represent the concurrent semantics of a graph grammar.

    We also extend previous works in occurrence graph grammars to include the notion of negative application conditions, which are commonly used in the modelling of systems as graph transformations systems nowadays.
  \item[Chapter~\ref{ch:tests}:] This chapter presents how we used the concurrent rules and occurrence graph grammars, discussed in previous chapters, to generate test cases using verigraph.
  \item[Chapter~\ref{ch:related-work}:] In this chapter, we show related works, focusing on the generation of test cases based on models and formal methods.
  \item[Chapter~\ref{ch:conclusions}:] This chapter summarizes our results and presents the conclusions. Moreover, it shows remaining open problems and future work.
  \item[Apendix~\ref{app:tutorial}:] This appendix contains the verigraph manual, explaining in the details how our tool can be installed and used.
\end{description}
